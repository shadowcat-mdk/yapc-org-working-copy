\documentclass{article}      % Specifies the document class

\newcommand{\pvar}[1]{\texttt{\$#1}} % perl variable, in typewriter font
\newcommand{\ptt}[1]{\texttt{#1}}    % typewriter font, for perl in text

                             % The preamble begins here.
\title{\ptt{Quantum::Entanglement} \linebreak Perl gets (quantum) physical}
\author{Alex Gough}
% \date{January 21, 1994}      % Deleting this command produces today's date.

\begin{document}             % End of preamble and beginning of text.

\maketitle                   % Produces the title.

\begin{abstract}
\noindent
Quantum mechanics is one of the stranger things to have emerged from
physics over the last hundred years.  It has led the way to new
understanding of a diverse range of fundamental physical phenomena
and, should recent developments prove fruitful, could also lead to an
entirely new mode of computation where previously intractable problems
find themselves open to easy solution.

The \ptt{Quantum::Entanglement} module attempts to ``port'' some of the
functionality of the universe into Perl.  Variables can be prepared in
a superposition of states, where they take many values at once, and
when observed during the course of a program, will collapse to have a
single value.  If variables interact then their fates are linked so
that when one is observed and forced to collapse the others will also
collapse at the moment of observation.

The module also makes possible simple complex logic gates, such as
$\sqrt{\neg}$, and allows for some algorithms from quantum
computing to be run, for instance Shor's method for factoring numbers.
As well as sensible programs the module also allows you to play
hangman against an opponent who cannot know the word you are trying to
guess, provide solvable Countdown maths puzzles and run subroutines in
multiple universes.
\end{abstract}

\section{Quantum Mechanics in a Nutshell}

\subsection{The Ghost in the Atom}

At the start of the last century, physics was in
trouble. The tattered edges of the theories which had served science
so well for so long were becoming exposed by increasingly accurate
experiments. The mainstays of thermodynamics, electromagnetism and
Newtonian mechanics could not explain, or more importantly
predict, the observed spectra of atoms and hot bodies.

Ever alert for the call to action, physicists soon assembled a new
version of mechanics, resting on a small set of simple assumptions and
equations which was to revolutionise our understanding of the
universe.  From this theory sprang startling results: physical
quantities such as energy were now able to take on only a set of
precisely defined values; particles like electrons were also waves
but waves like light were also particles, systems could exist in
a ghostly overlay of many different states at once, but only if you
didn't look at them and cats the world over feared for their
existential well-being.

Hyperbole aside quantum theory is not quite as complicated as it
sometimes seems.  At its root is a simple equation and a few rules
which must be applied when calculating physical outcomes.  The rest of
this section presents a brief overview of this structure, do not worry
if you cannot follow, just skip ahead to the perl.

\subsection{Solving an Equation}
\label{sec:equations}
{\noindent
\begin{picture}(0,0)
\put(-15,0){\emph{You can safely skip this section\ldots}}
\end{picture}
}

Classically, a particle can be thought of as being a blob of stuff,
with a precisely defined set of values associated with its position,
direction of motion, speed, energy and so on.  In quantum mechanics, this
version of events is overturned.  A particle instead becomes a wave,
or curve, which satisfies an equation, its position and energy must be
extracted from this equation via a bit of mathematical trickery which
is outlined below.

Because physics is mostly concerned with the energy a particle can
have, we usually start by finding sets of
functions which satisfy Schr\"{o}dinger's famous equation,
\begin{equation}
	[-\hbar^2\frac{\partial^2}{\partial x^2}
	 + \mathcal{V} ]\phi_n(x) = E_n\phi_n(x) \label{eq:schrod}
\end{equation}
where $E_n$ must be a real number, corresponding to the energy of a
particle in the state $\phi_n(x)$.  We call $E_n$ an eigenvalue, and
$\phi_n(x)$ an eigenfunction of the equation. Generally $\phi_n(x)$ will be
something like $\sin(nx/a)$ or $\cos(nx/a)$, it is also this equation
which forces $E_n$ to take quantised values which depend on the
particular situation we are looking at.
We then ask what $\phi_n(x)$ actually \emph{represents}.
We find that if we square this function, then if we know that a
particle has an energy of $E_i$ then if we look at where the particle
is, we will see it at a position $x=k$ with a probability of
$|\phi_i(k)|^2$.

Sometimes things get a little bit more complicated though.  What if we
want to know what the momentum of the particle is, rather than its
energy? Instead of finding solutions to
equation~\ref{eq:schrod} we need to find solutions to another equation
\begin{equation}
	-i\hbar\frac{\partial}{\partial x}\phi'_m(x)
		= P_m\phi'_m(x) \label{eq:mom}
\end{equation}
which could be different from the solutions we found to
equation~\ref{eq:schrod}.  But what if we want to know both?  In
general it turns out that we cannot, but we can measure one (the
momentum say) for which we get a value $P_k$, which tells us that the
particle must have a
probability distribution described by $|\phi'_k(x)|^2$.  When we then
measure the energy of the particle, we cannot be
sure of the value we will get.  We must add together possible
solutions to equation~\ref{eq:schrod} (using lots of different values
of $n$) so that the curve we get is the same as $\phi'_k(x)$, or
\begin{equation}
\phi'_k(x) = a_0\phi_0(x) + a_1\phi_1(x) + a_2\phi_2(x) + \ldots
\label{eq:super}.
\end{equation}
In words, we form a superposition of possible $\phi_n(x)$.  When we
then measure the energy of our particle, we get $E_0$ or $E_1$ and so
on, each with a probability given by $a_0^2$ or $a_1^2$ (so long as
$\sum_{i=0}^{\infty} a_i^2 = 1$).  Of course, once we've measured the
energy, we again know exactly which energy state the particle is in,
although we will no longer be able to assign a unique value to the
momentum of the particle.

{\noindent
\begin{picture}(0,0)
\put(-15,0){\emph{Whoa, you might want to read this last bit.}}
\end{picture}
}

When in a superposition like equation~\ref{eq:super} we cannot treat
the particle as if it were in a particular state, but instead must
think of it as occupying every possible state in the equation at the
same time.  Only when we observe some property of the particle does it
take on a single, well defined state.

Or, put more briefly, sometimes a quantity can have many possible
values at once, the chance of it collapsing to have a one particular
value is given by the square of a coeficient in a superposition
equation.  Even more briefly: things are strange.

\section{Creating Superposed States}

It is quite hard to provide a complete version of the above in perl,
so we need to make some compromises.  Instead of solving thousands of
equations each time we want to do something, we will assume that only
one set of values (or solutions) exist which will be provided by the
user, we will also forget entirely about
eigenfunctions (the $\phi_n(x)$ above).  This still allows us to give
variables multiple values at the same time, and to control which values
they collapse to when observed by using probability amplitudes.

This means that we can create a variable in the following
superposition of states
\begin{equation}
	\pvar{var} =    a_0|0 \rangle +
			a_1|1 \rangle +
			a_2|2 \rangle,
\label{eq:borw}
\end{equation}
(where the $| \ldots \rangle$ pattern indicates a possible value of
the variable) we can then use \pvar{var} throughout a program as we
would any normal variable, but so long as we do not try to find out
the actual value of \pvar{var} every possible value it can take will
be present in every calculation, this also means that the results of
any calculations involving the variable will have many possible values.

\subsection{The \ptt{entangle()} function}

The \ptt{Quantum::Entanglement} module adds an \ptt{entangle()}
function to perl, this takes a list of amplitudes and values and
returns a variable in a superposition of values, saying
\begin{verbatim}
  $var = entangle( $a0 => 0,
                   $a1 => 1,
                   $a2 => 2 );
\end{verbatim} %$
creates the superposition of values in~\ref{eq:borw} above.

\subsection{Observation and Collapse in Perl}

We then need to decide what happens when we observe our variable, and
what exactly we mean by \emph{observe}?  Taking a broad
definition of observation as being ``anything that reveals the values
which a variable has'' seems about right.  Perl
provides us with many ways of doing this, there are the obvious acts
of printing out a variable or testing it for truth, but also operators
such as \ptt{eq} or \ptt{>=} tell us something.

How do we decide which values a variable collapses to when observed?
Well, each possible value has an associated probability amplitude, so
all we need to do is build up a list of possible outcomes, add up the
amplitudes, square the result, then use this to bias the value (or
values) that the variable collapses to.

For the superposition in \ref{eq:borw}, if we were to say
\begin{verbatim}
 if ($var == 1) {
   print 'equal to 1';
 }
 else {
   print 'not equal to 1';
 }
\end{verbatim} %$
we would see `\ptt{equal to 1}' printed with a probability of
	$a_1^2 / (a_1^2 + (a_0 + a_2)^2)$
and `\ptt{not equal to 1}' with a probability of
	$(a_0+a_2)^2 / (a_1^2 + (a_0 + a_2)^2)$.

Now an interesting
possibility arises, what if $a_0 = -a_2$, can we do this?  The answer
is ``yes'', and this means that the collapse of our variables in
different circumstances will be different, if we had just wanted to
\ptt{print} our variables, then there is a chance that either value
will be output, but if we test our superposition for ``oneness''
we will find that it is \emph{always} the case.

\subsection{Hangman}
\label{sec:hangman}
We can now start to write interesting programs, even before we hit the
really weird stuff.  You have probably heard of the game ``twenty
questions'', where one player thinks of an object, and must answer
`yes' or `no' to twenty questions  from the other before he has to
guess what the object is.  If you
are careful, it is possible to play this without actually thinking of
an object, answering the questions any way you want so long as you can
eventually think of an object which is consistant with your answers at
the end of the game (some people might call this cheating though).
The following program allows you to play hangman against an opponent
who does not know which word you are trying to guess, until that is,
you have made some guesses.

First of all, we read in a list of words of a given length (to make
the game a little bit easier)
\begin{verbatim}
#!/usr/bin/perl -w
use Quantum::Entanglement;
my @words;
print "Reading word list...";
open DICT, '</usr/share/dict/words' or die "dict: $!";
while (<DICT>) {
  chomp;
  push(@words,1,lc($_)) if length($_) == $ARGV[0];
}
close DICT; print "done\n";
\end{verbatim}
you may notice that the array \ptt{@words} has two elements added for
each word, the first is the probability amplitude we want to assign
the word, and the second is the word itself, this allows us to create
a single word which is a superposition of all possible words by saying
\begin{verbatim}
my $word = entangle(@words);
\end{verbatim} %$
we then ask our player to guess a letter (or range of letters
with \ptt{[a-e]}) as well as providing a way to give up,
\begin{verbatim}
while (<STDIN>) {
  chomp;
  print "So, you give up... $word\n" and last if /tell/;
\end{verbatim} %$
If the player gives up when there are still many possible values of
the word remaining, printing out \pvar{word} will cause a collapse to
a single word.
If they want to continue playing, we test our word with a regular
expression\footnote{As regex
binding cannot be overloaded, we need to cheat a bit.}, this will
select two groups of possible results, those words which match our
guess, and those that do not, and will then select one of them at
random causing \pvar{word} to collapse so that it can \emph{only} be
those which match (or not, depending on the direction of collapse),
this all happens transparently.
\begin{verbatim}
  if (p_op($word, '=~', qr/$_/)) {
    print " \$word does contain $_\n";
  }
  else {
    print " \$word does not contain $_\n";
  }
}
\end{verbatim} %$

\section{Entanglement and non-local Collapse}

Another result to emerge from the equations of quantum theory was that
when two quantum particles (such as photons) interacted, their
wavefunctions (the $\phi_n(x)$) became linked with a single
wavefunction describing the entire system ($\phi_{n,m}(x_1,x_2)$).
This connection meant
that when one of the particles was measured and collapsed to some
definite state, the other would also collapse, even if separated by great
distances.  This manifests itself as a greater degree of correlation
between entangled particles than would be expected if the two
particles collapsed at different times (the Bell inequality) or did
not become entangled when they interacted.

\subsection{Entanglement in Perl}

Whenever superposed variables interact, or are involved in
calculations, the results of these as well as the variables
themselves become entangled.  This means that they will all collapse
at the same time, so as to remain consistant with their history.  For
instance, the following
\begin{verbatim}
 $var = entangle( 1 => 0, 1 => 1 );
 $result = $var * 2;
 if ($result == 2) {
   print "\$var equals 1\n" if $var == 1;
 } else {
   print "\$var equals 0\n" if $var == 0;
 }
\end{verbatim}
will cause \pvar{var} to always collapse to 1 if \pvar{result} was 2, or 0 if
\pvar{result} was not equal to 2.  Although we cannot predict which
way \pvar{result} will collapse, once it has, we can predict what
\pvar{var} will be because of entanglement.

\section{Simple Complex Logic}

If entanglement isn't enough to scare you, quantum theory also makes
possible new logical operations, first though, we need to expand the
what we can do with a superposition.

\subsection{Complex Amplitudes}

If we can have negative values as the coefficients of our
superpositions, it seems sensible that we can also use complex
numbers.  The only difference being that instead of just squaring the
number when working out our probability, we need to square the
\emph{size} of the number.  (eg. $|1+2i|^2 = 5 =|1-2i|^2$.)

\subsection{Generic Logic Gate}

\begin{figure}
\begin{center}
\begin{picture}(160,80)(0,10)
\put(40,10){\framebox(80,80){}}

\put(20,20){\line(1,0){30}} % io lines
\put(20,80){\line(1,0){30}}
\put(110,20){\line(1,0){30}}
\put(110,80){\line(1,0){30}}

\put(50,20){\circle{4}} % circles on end of lines
\put(50,80){\circle{4}}
\put(110,20){\circle{4}}
\put(110,80){\circle{4}}

\put(50,20){\line(1,1){60}} % path lines
\put(50,80){\line(1,-1){60}}
\put(50,20){\line(1,0){60}}
\put(50,80){\line(1,0){60}}

\put(0,50){Input}	% input and outputs
\put(140,50){Output}

\put(10,15){\huge{0}} 
\put(10,75){\huge{1}}
\put(140,15){\huge{0}}
\put(140,75){\huge{1}}

\put(70,24){\large{$a_{00}$}} % coefficients
\put(70,72){\large{$a_{11}$}}
\put(95,37){\large{$a_{10}$}}
\put(95,60){\large{$a_{01}$}}
\end{picture}

\caption{A generic logic gate, capable of accepting \emph{one} input,
which can take either of two possible values.
The coefficients $a_{ij}$ indicate the
probability that a given input state will be transformed into a given
output state.}
\label{fig:logicgate}
\end{center}
\end{figure}

Figure~\ref{fig:logicgate} shows a generic logic gate which takes
a single input, and transforms it into a single output.  The
coefficients $a_{ij}$ give the probability that a given input will be
transformed into a given output.  If $a_{00} = a_{11} = 1$ and
$a_{01} = a_{10} = 0$ then the gate does nothing (a pass-though), if
$a_{01} = a_{10} = 1$ and $a_{00} = a_{11} = 0$ then the gate will
invert its input (a NOT gate).  We can also allow $a_{ij} = 0.5$,
which will act as a randomising device producing an output stream
which is not correlated with its input.  If we chain two of these
together, so that the input of one becomes the output of the second,
then we continue to get perfectly random noise.

\subsection{Complex Logic Gate}

Things become interesting if instead of using real numbers and
definite inputs (a classical
picture), we allow $a_{ij}$ to be complex numbers, let the input
and output be a superposition of
input states and modify the workings of our gate so that the
probabilities ($a_{ij}$) multiply the coefficients of our
superposition equation, so that an input state of
\[
	b_0|0\rangle + b_1|1\rangle
\]
becomes an output state of
\[
	(a_{00}b_0 + a_{10}b_1)|0\rangle +
	(a_{11}b_1 + a_{01}b_0)|1\rangle.
\]
Now, if we let $a_{00} = a_{11} = i/\sqrt{2} $ and
$a_{01} = a_{10} = 1/\sqrt{2}$, an input state of $|0\rangle$
will be transformed into a superposition of two output states
\[
	\frac{i}{\sqrt{2}}|0\rangle + \frac{1}{\sqrt{2}}|1\rangle
\]
which if observed, will collapse with equal likelyhood to either 0 or
1 (as the size of the coefficients for each state is the same).  If we
use only one of these gates, then we see the same random noise as we
did with the classical random gate above.

What if we restrain ourselves and rather than looking at the output of
the first gate, send the superposed state though an identical
gate? We find that the output of the second gate will be in the state
\[
	\left( \frac{i}{\sqrt{2}}\frac{i}{\sqrt{2}}
	+ \frac{1}{\sqrt{2}}\frac{1}{\sqrt{2}} \right)| 0 \rangle
       +\left( \frac{i}{\sqrt{2}}\frac{1}{\sqrt{2}}
	+ \frac{1}{\sqrt{2}}\frac{i}{\sqrt{2}} \right)| 1 \rangle
\]
which simplifies to
\[
	\left(-\frac{1}{2}+\frac{1}{2}\right)|0\rangle
     +	\left(\frac{i}{2} +\frac{i}{2}\right)|1\rangle
		= 0|0\rangle + i|1\rangle
\]
if the output of the second gate is observed, then as only one
state has a non-zero coefficient, it \emph{must} collapse to 1.
Whereas the classical version of this gate had the same effect when
two were chained together, the quantum version produces noise if one
is used, or negates its input if two are chained together, in other
words, we have a $\sqrt{\neg}$ (square root of NOT) gate, a
function entirely absent from classical computing.

\[
	\sqrt{\neg}(0) = |\textit{noise}\rangle
\]
\[
	\sqrt{\neg}(|\textit{noise}\rangle) = 1
\]

\subsection{$\sqrt{\neg}$ in Perl}

The \ptt{Quantum::Entanglement} module allows subroutines to create new
states based on the current set of states by using the function
\ptt{q\_logic} and a subroutine.  The following program implements a
$\sqrt{\neg}$ gate:

\begin{verbatim}
#!/usr/bin/perl -w
use Quantum::Entanglement qw(:DEFAULT :complex);

my $foo = entangle(1 => 0);
$foo = q_logic(\&root_not, $foo);
$foo = q_logic(\&root_not, $foo);

print "\$foo is true!\n" if $foo;

sub root_not {
  my ($prob, $val) = @_;
  return( $prob * i / sqrt(2) , $val,
	      $prob / sqrt(2) , !$val ? 1 : 0);
}
\end{verbatim} %$

The output is always \ptt{\$foo is true!}.

\subsection{$\sqrt{\neg}$ in the Real World?}

The $\sqrt{\neg}$ gate is more than a mathematical sleight
of hand as devices exist which perform this function.  The semi-silvered
(or two-way) mirrors, much beloved of gangsters and politicians for
their car windows, can be designed so that a photon has a fifty-fifty
chance of passing through the mirror or being reflected.  Quantum
physically, this means that a photon follows a superposition of
\emph{both} possible paths (until observed).  When the
light leaving the mirror is carefully bent back on itself (see
figure~\ref{fig:mirror}) and sent through another two-way mirror the
two superposed paths are recombined causing the photon to emerge from
the second mirror in one direction only.  If the path taken by the
photon is determined (even when inferred by its absence from another
path) it can emerge from the final mirror in \textit{either} direction.
This can only be explained
if it is assumed that the photon follows both paths through the
equipment, and only if the route it takes is not examined.

\begin{figure} % mirrors as root not
\begin{center}
\begin{picture}(100,100)

\put(3,93){In}
\thinlines
\put(20,80){\line(1,-1){20}} 	% 1/2 mirrors
\put(60,40){\line(1,-1){20}}

\put(0,70){\vector(1,0){20}} 	% inputs
\put(30,100){\vector(0,-1){20}}
\put(33,90){0}
\put(5,75){1}

\put(80,30){\vector(1,0){20}}
\put(70,20){\vector(0,-1){20}}
\put(85,33){1}
\put(73,8){0}
\put(87,5){Out}

\thicklines                  	% full mirrors
\put(20,40){\line(1,-1){20}}
\put(60,80){\line(1,-1){20}}
\thinlines

\put(33,30){\line(1,0){5}}	% dashed lines
\put(43,30){\line(1,0){5}}
\put(53,30){\line(1,0){5}}
\put(63,30){\line(1,0){5}}

\put(33,70){\line(1,0){5}}
\put(43,70){\line(1,0){5}}
\put(53,70){\line(1,0){5}}
\put(63,70){\line(1,0){5}}

\put(30,33){\line(0,1){5}}
\put(30,43){\line(0,1){5}}
\put(30,53){\line(0,1){5}}
\put(30,63){\line(0,1){5}}

\put(70,33){\line(0,1){5}}
\put(70,43){\line(0,1){5}}
\put(70,53){\line(0,1){5}}
\put(70,63){\line(0,1){5}}

\end{picture}
\end{center}
\caption{Two semi-silvered mirrors acting as $\sqrt{\neg}$ gates, a
single photon passing through the system takes both paths (dotted
lines) simultaneously}
\label{fig:mirror}
\end{figure}

\section{More Useful Algorithms}

The fundamental links between quantum and classical computing
give us new insights into what is and isn't computable (the Turing
machine functions are a subset of the functions of the universal
quantum computer) which is certainly interesting physics and
comforting for computer scientists, but is
quantum computing actually useful for anything?  About ten years after
the subject had its genesis, algorithms were discovered which could
outperform their classical counterparts.  This translates well into
research funding so expect to hear more in the future.

\subsection{Factoring Numbers}

Factoring large numbers is well known as a hard problem, it is also an
area which is begging for a more efficient algorithm than classical
brute-force methods can manage as many widely used encryption schemes
gain their security from the difficulty of discovering two
large prime numbers given only their product.

\subsubsection{Shor's Algorithm}

Back in 1995, Peter Shor discovered an algorithm which was able to
factor large numbers.  By abusing the periodic nature of certain
mathematical functions, coupled with Fourier transforms and the parallelism
which can be obtained from superposed states, it is possible to factor
numbers in nearly constant time.

\subsubsection{Shor's Algorithm in Perl}

\label{sect:shor}

The \ptt{Quantum::Entanglement} module can run Shor's algorithm.  An
example of this, along with a short explanation, is included in the
distribution of the module from the CPAN.  It does not
quite achieve the desired speed improvement but this is
the price of simulation.  The following is an attempt to factor 21
using a PII-400 and the zsh \ptt{time} builtin:
\begin{verbatim}
lintilla% time ./shor.pl 21 
Performing initial classical steps:
Using q:512, x:4
Starting quantum steps
Performing F = x**|a> mod n
$register2 collapsed to 1
Finding period of F (this is where you wish for a QCD)
Period of F = x**|a> mod n is 2
3 * 5 and 21 might share a gcd (classical step)
3, 1 could be factors of 21
./shor.pl 21  125.67s user 0.04s system 89% cpu 2:20.14 total
\end{verbatim} %$

\section{Breaking the Laws of Physics}
\subsection{Controlling Collapse}

Given that we have a framework which closely mirrors some of the laws
of physics, there seems to be no good reason why we might not
sometimes want to override these rules.  We may occasionally want to
force a superposition to collapse to a given value, perhaps when
preparing it for use in a more complex problem, we might also want to
cheat at hangman.

The module provides a flag \pvar{::conform} which,
when set to a true value, will cause any operation which could return
either true or false values, to always return truth, if possible, and
retain only those states which cause the true value. Variables which
are entangled with the one which collapsed also collapse as before.
A method of saving the state of a superposition is also available
so that it can be restored after a collapse. This can be used on
multiple variables so that the entangled relationships between them
are retained.

\subsection{Playing Perfect Hangman}

If the program listed in section~\ref{sec:hangman} is modified so that
it contains
\begin{verbatim}
 $Quantum::Entanglement::conform = 1;
\end{verbatim} %$
near the start, this will cause the collapse of \pvar{word} to always
follow the guesses of the player, if at all possible.  On the other
hand, by changing the test we use on \pvar{word} to \verb|'!~'| then
we find ourselves playing against a dealer who tries his best to make
our guesses fail.

\section{Internal Details}

\begin{quote}``Imagine a box, a really big, really black box, in a
black room, with no lights, and maybe no box.''\end{quote}

Writing this behavior into Perl presents an interesting challenge, a
means of representing a superposition is required, as is some way of
allowing different variables to know about each other, without
creating circular references which would stand in the way of garbage
collection.  We also need a means to cause collapse, as well as a
robust mechanism for dealing with both real and complex numbers.
Thankfully Perl provides a rich set of ingredients which can more than
satisfy these requirements without making the job so hard that it becomes
impossible.

\subsection{Overloading}

Perl has a fairly good overloading mechanism, it allows the working of
most (although not all) operators, and some core functions, to be
redefined for objects \ptt{bless}ed into a given package.  Each
entangled variable is simply an object, for which all operators have
been modified so as to act in parallel on the values it can take.
Collapse is caused by any operator which identifies some properties of
the superposition, \verb+==+, or stringification say.  Operators which
do not reveal anything about the state of a variable return a new
entangled variable.

\subsection{Representation}

If we deal with the simple case of two variables, which are entangled
with one another after some sort of interaction, we find that for each
possible value that the first ($a_1$ say) can take, the second can
take each of its values ($b_j$), the same being true for $a_2$ and so
on.  This can be displayed as shown in figure~\ref{fig:twovals}.  This
is nothing more than a two dimensional table, so is natural to use
a list of lists.  We have left out the probability amplitudes, but
these can just be slipped in the column next to the values with which they are
associated.

\subsection{Interactions and Entanglement}

When two entangled variables interact (usually producing an entangled
result), we need to join together both those variables, and all the
variables which have interacted with those variables earlier in the
program.  This is acheived by creating a new table of states, which is
the `cross' of the two previous sets of states.  This is illustrated
in figure~\ref{fig:twovals}.

\begin{figure}
\begin{center}

\begin{tabular}{ccc}
	\ptt{\$a = entangle(\ldots)}
	&
	\ptt{\$b = entangle(\ldots)}
	&
	\ptt{\$c = \$a * \$b}
\\
	$ \begin{array}{c}
	a_0 \\
	a_1
	\end{array} $
	&
	$ \begin{array}{c}
	b_0 \\
	b_1
	\end{array} $
	&
	$ \begin{array}{ccc}
	a_0b_0 & a_0 & b_0 \\
	a_0b_1 & a_0 & b_1 \\
	a_1b_0 & a_1 & b_0 \\
	a_1b_1 & a_1 & b_1
	\end{array} $
\end{tabular}

\end{center}
\caption{Before any interaction, the possible values of $a$ and $b$
are uncorrelated, afterwards, each value of $a$ must be paired
with every possible value of $b$, values of $c$ then depend on these pairs}
\label{fig:twovals}
\end{figure}

\subsection{Collapse}

Using this tabular approach it becomes easy to remove unwanted states
when variables are observed so that all variables are modified at the
same time.  If we imagine that we tested \pvar{b} above, and found
that it was equal to $b_1$, we simply remove all the rows
in which it isn't equal to $b_1$ as shown in figure~\ref{fig:destroy}.

\begin{figure}
\begin{center}
\begin{tabular}{c}
	\ptt{\$b == b1} \\
	$ \begin{array}{ccc}
	a_0b_0 & a_0 & b_0
\begin{picture}(0,0)
	\put(0,2){\line(-1,0){60}}
\end{picture}
\\
	a_0b_1 & a_0 & b_1 \\
	a_1b_0 & a_1 & b_0
\begin{picture}(0,0)
	\put(0,2){\line(-1,0){60}}
\end{picture}
\\
	a_1b_1 & a_1 & b_1
	\end{array} $
\end{tabular}
\end{center}
\caption{Unwanted states are easy to remove, as we must only delete
rows from a table, automagically affecting all variables.}
\label{fig:destroy}
\end{figure}

\subsection{Indirection}

Having built up this table, how do we know which data belongs to which
variable, and how do we ensure that we can delete unwanted data?
This problem is solved by using two levels of references, so that each
variable contains a reference to a reference to a list of lists, each
variable also needs to know the position its data has in this table,
but also needs to know the positions of every other variable, so that
when two of these tables are joined together, each variable has its
position updated at the same time.
This is acheived through each variable also holding a reference to a
list of references to these offsets.

By using two levels of referencing, it becomes easy to redirect the
target of the middle reference, as happens when two entangled
variables interact, requiring that they share the same table of
outcomes.  When a new variable is created, it has this structure
\begin{verbatim}
  my $var = [\$universe,1,\$offsets];
  $offsets->[0] = \ $var->[1];
\end{verbatim} %$
where \pvar{universe} is the same for all variables with which it has
interacted.  This means that changing the target of \verb+${$var->[0]}+ to a
new universe, will change the target for all variables inhabiting the old
universe. The same
holds for the array \ptt{@\$offsets}.  This structure also avoids any
circular referencing, which means that variables are destroyed
sensibly as nothing outside the program using the module stops
variables from going out of scope.

\subsection{Limitations in Perl}

While Perl's referencing, overloading and easy creation and extension
of complicated data structures makes modules like this easy to
write, it still lacks features which could make the job even easier.

It is not currently possible to define new operators and
some key operations (such as \verb+=~+) cannot be overloaded.  A few of
the automatically generated overloaded operators cannot be
distinguished from one another within the module doing the
overloading, which makes it hard to producing warnings consistantly,
and can lead to ugly code elsewhere.

Perl lacks `friendly' multi-dimensional data structures, making it hard to
take slices along two different dimensions of a matrix without writing
excess code to control doing so.  It is also not possible to easily
define array slices with constant steps (\ptt{2,4,6,8\ldots}). On
balance though these are small complaints and mostly the price of
the very general approach of Perl.

\section{Into the Future \ldots}

The following section details some potential improvements which might
find their way into the module in the future.  Each tries to
overcome current limitations or extend the range of
phenomena simulated by the module.

\subsection{Delaying Collapse}

As it stands the observer which causes the collapse of variables in
any program is the program itself.  This misses the fun which
could be had by making the person running the program the observer,
effectively keeping everything inside the program in a
superposition of states until some output is produced.  How to deal
with situations like
\begin{verbatim}
  $bar = entangle(1=>0, 1=>1);
  if ($bar == 0) {   # no output produced
    $foo = $bar * 2;
  }
  else {             # output produced
    print "I think \$bar != 0\n";
    $foo = $bar + 1;
  }
  print "$foo\n";
\end{verbatim} %$
poses a little problem.  Do we run the entire program, capturing
all potential output, then at the end collapse and dump one of the
possible versions or do we assume that any conditional branching which
causes different output to be produced always acts as instant
observation?

\subsection{Parallel Execution}

The point above draws us towards parallel execution.  It would
be nice to play about with a source filter, and some magic, to allow
all possible execution paths to be followed.  Perhaps allowing them to
interfere with each other, as happens with the real $\sqrt{\neg}$ gate
in figure~\ref{fig:mirror}.  This could also allow for execution to be
spread over multiple processors or machines, although Perl is not well
suited to HPC applications.

\subsection{Uncertainty}

Recalling the discussion in section~\ref{sec:equations}, in the real
world (which is weirder than we can be) the possible values which a
measurement can give depend on what else has been measured about a
particle.  For instance, we saw that we cannot know both the energy
($E_n$) and momentum ($P_i$) of a particle.  This known as the
\emph{uncertainty principle} and it would be lovely if it could be
added to Perl.

The fully fledged functional version is a little beyond Perl, but
by using matrix operators, each of which have different sets of
eigenvalues, it should be possible.

\subsection{It's Physics Jim, but not as we know it}

This module does fall short of physical reality in a few important
areas, some of which are listed below:

\begin{itemize}
\item \textbf{No eigenfunction-like behaviour} All operators share the same set
of eigenfunctions, in real QM this is sometimes not the case, so
observing one thing would cause some other thing (even if already
observed) to fall into a superposition of states again.

\item \textbf{Certain observables cannot simultaneously have precisely
defined values} This follows from the point above.  The famous uncertainty
principle arises as position and momentum have different
sets of eigenfunctions.  In this module, it is always possible to collapse
the system so that a value is known for every entangled variable.

\item \textbf{Perl is not a quantum computing device}
Perl, alas, is currently only implemented on classical computers, this
has the disadvantage that any quantum algorithm will run in
exponential time.  This might be remedied in future releases of Perl.
Just not anytime soon.

\item \textbf{Quantum information cannot be copied with perfect fidelity}
It is impossible to perfectly clone a real entangled state without
`damaging' in some way either the original or the copy.  In this
module, it is possible as we have special access to the states of our
variables.

\item \textbf{Cannot generate perfectly random numbers}
It is well known that classical computers cannot produce a perfectly
random sequence of numbers, as this module runs on one of these, it
also suffers the same fate.  It is possible to give a classical computer
access to a perfect random number generator though (essentially by
linking it to a suitable physical system) in which case this is no
longer a problem.

\end{itemize}

\section{Just another thing}
\begin{verbatim}
#!/usr/bin/perl
use Quantum::Entanglement qw(:DEFAULT :complex);
$language = entangle(1,'python',1/i,'C',i*i,'perl',
                     1/i**2,'Java',i**5,'C#');
print "Just another $language hacker,\n" if $language eq 'perl';
\end{verbatim} %$

\section{More Information}

The 1985 paper ``Quantum theory, the Church-Turing principle and the
universal quantum computer''
\footnote{\ptt{http://www.qubit.org/resource/deutsch85.ps}} by David
Deutsch which `launched' quantum computing is available from the
Oxford Center for Quantum Computation, the rest of their site has good
number of introductory articles.  Another paper worth reading is
``Machines, Logic and Quantum
Physics''\footnote{\ptt{http://xxx.lanl.gov/abs/math.HO/9911150}} by
David Deutsch, Artur Ekert and Rossella Lupacchini.

The module distribution (available from the CPAN) comes with a few
short demonstrations of what the module can do, along with plenty of
explanation (including Shor's algorithm from
section~\ref{sect:shor}). If you do anything interesting with the
module, or have any questions, do feel free to contact me:
\ptt{alex@rcon.org}.

\end{document}